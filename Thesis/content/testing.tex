Testing is an important part of the software development process, even more so in case
of software handling sensitive data. The core of our implementation, however, is based
on using external hardware devices, which were specifically designed to confirm every
operation with the user before performing it. That makes it impossible to write fully
automated tests. Writing automated tests only for the parts that do not directly interact
with the authenticator, or mocking\footnote{A technique of replacing one part of real
implementation with a "fake" version during testing, with the goal of making another
part easier to test.} the functions that perform the communication makes little sense
because such tests would not cover the most critical parts of the code.

For these reasons, we include test cases for manual testing, which can be easily
performed by any user of the plugin. These test cases are written in such a way that
they can also serve as a user manual.

All cases assume a default configuration of Windows 10, version 1903 or higher (mainly, default \glsxtrshort{uac} settings),
default KeePass configuration, and a single authenticator connected via USB. All cases start with KeePass running and no database open.

Note that at this time, a single authenticator can only hold a key for one database.
Adding a key for a new database (as done in some of these tests) overwrites any other
key previously stored on the authenticator.

\setlist[1]{before={\parskip=\docparskip},after={\vspace{-\docparskip}}}

\section{Positive tests}\label{sec:positive-tests}

This section describes test cases for the intended usage scenarios.
Note that these cases are written for a database protected by a master password. To test with a database
protected by a key file, the same steps apply, except any instruction to enter the password
is changed to choose the key file.

\subsection{Add a key for a new database}\label{subsec:add-a-key-for-a-new-database}

\begin{enumerate}
	\item Create a new KeePass database.
	\item In the "Create Composite Master Key" dialogue, fill in "123" as master password.
	\item Go to Tools $\rightarrow$ Options $\rightarrow$ KeePassFIDO2 and click "Add a new FIDO key".
	\item Enter your authenticator PIN when prompted (if configured), and click "Submit".
	\item Confirm the UAC prompt for "DeviceCommunicator.exe".
	\item Perform user presence/verification check as requested by the authenticator.
\end{enumerate}

\textbf{Expected result:}
A message confirming the key was added appears in the "KeePassFIDO2 Options" window.

\subsection{Unlock a database using the authenticator}\label{subsec:unlock-a-database-using-the-authenticator}

\begin{enumerate}
	\item Open the database created in \autoref{subsec:add-a-key-for-a-new-database}.
	\item In the "Open Database" dialogue, select "Key File: FIDO2 Key Provider" as unlock method and click "OK".
	\item Enter your authenticator PIN when prompted (if configured), and click "Submit".
	\item Confirm the UAC prompt for "DeviceCommunicator.exe".
	\item Perform user presence/verification check as requested by the authenticator.
\end{enumerate}

\textbf{Expected result:}
The database was unlocked.

\subsection{Unlock a database using the original method}\label{subsec:unlock-a-database-using-the-original-method}

\begin{enumerate}
	\item Open the database created in \autoref{subsec:add-a-key-for-a-new-database}.
	\item In the "Open Database" dialogue, select "Master Password", type "123" and click "OK".
\end{enumerate}

\textbf{Expected result:}
The database was unlocked.

\section{Negative tests}\label{sec:negative-tests}

This section describes steps to test that common error situations are correctly handled.

\subsection{No database open}\label{subsec:no-database-open}

\begin{enumerate}
	\item With no database open, go to Tools $\rightarrow$ Options $\rightarrow$ KeePassFIDO2 and click "Add a new FIDO key".
\end{enumerate}

\textbf{Expected result:}
An error message is shown, saying that a database needs to be open first.

\subsection{Error while adding a new key}\label{subsec:error-while-adding-a-new-key}

\begin{enumerate}
	\item Open the database created in \autoref{subsec:add-a-key-for-a-new-database}.
	\item In the "Open Database" dialogue, select "Master Password", type "123" and click "OK".
	\item Go to Tools $\rightarrow$ Options $\rightarrow$ KeePassFIDO2 and click "Add a new FIDO key".
	\item Disconnect the authenticator from the computer.
	\item Submit the "FIDO2 PIN" dialogue.
	\item Confirm the UAC prompt for "DeviceCommunicator.exe".
\end{enumerate}

\textbf{Expected result:}
An error message is shown, including a device communicator exit code.


\subsection{Error while unlocking the database}\label{subsec:error-while-unlocking-the-database}

\begin{enumerate}
	\item Open the database created in \autoref{subsec:add-a-key-for-a-new-database}.
	\item In the "Open Database" dialogue, select "Key File: FIDO2 Key Provider" as unlock method and click "OK".
	\item Disconnect the authenticator from the computer.
	\item Submit the "FIDO2 PIN" dialogue.
	\item Confirm the UAC prompt for "DeviceCommunicator.exe".
\end{enumerate}

\textbf{Expected result:}
An error message is shown, including a device communicator exit code.
