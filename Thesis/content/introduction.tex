Passwords have been in use as a primary method of authentication in computer systems for decades.
They are a well-understood concept by both the engineers implementing those systems and by the end-users.

Still, they come with several problems.
With the computational power of today's hardware, they need to be long and complex to be secure,
which makes them hard to remember and annoying to type.
The problem with remembering becomes significantly worse upon the realization that an average person may use tens of different services,
each of which should be protected by a different password.
Of course, it is almost impossible to remember that many passwords, which is why most people tend to use just one, or a few passwords everywhere.
This practice presents a significant security risk, because compromising just one system may lead to compromising all other systems where the same password
is used by the given user and could be seen as one of the main arguments for moving to a different form of authentication.

There have been many attempts to find a suitable replacement over the years and predictions of passwords going away.
For example, Bill Gates made such a prediction already in 2004, and in 2011, IBM predicted that "you will never need a password again" within five years.
Yet, it is 2020, and passwords are still the primary method of authentication in most systems.
None of the possible replacements, such as digital certificates, one-time login links, biometrics, or single sign-on systems,
managed to combine the relative ease-of-use, security, cost-effectiveness, and ease of implementation and deployment.

Since passwords have been in wide use for a long time, solutions that aim not to replace them,
but rather remove some of their problems, were developed. Password managers\textemdash applications designed
to securely store all user's passwords in an encrypted form\textemdash solve the issue with passwords
reuse by removing the need to remember them, and may even provide better user experience by automatically filling in the correct credentials in most environments.
They are usually protected by another password themselves\textemdash a master password\textemdash which is the only password the user needs to remember.

The idea of replacing passwords altogether has not been lost, though,
and one of the most recent news in this area is FIDO2. A project by the FIDO Alliance\textemdash an open industry association with members including
many large technology companies, such as Amazon, Apple, Google, or Microsoft\textemdash and World Wide Web Consortium\textemdash
an organization responsible for creating standards for the World Wide Web.

FIDO2 uses authenticators\textemdash small devices similar to flash keys\textemdash
which are able to securely generate, store, and later find and use the correct credentials for each service,
while completely hiding the technical aspects from the end-users.

This approach provides several advantages over passwords, including complete elimination of phishing attack,
while keeping a great level of usability\textemdash from the users' perspective, using the device involves only connecting it to the computer and unlocking it with a short PIN when prompted.

The current issue with FIDO2 is that being a very new technology\textemdash de\-sig\-na\-ted as an official web standard in March 2019\textemdash even if it becomes successful,
it will take years before the majority of online services supports it and before the whole ecosystem around it develops.

Nevertheless, having a large list of big organizations behind it, and being focused on
all the aforementioned problems of alternatives (ease-of-use, security, cost-effectiveness, and ease deployment),
it has great potential.

The goal of this thesis is, therefore, exploring the possibilities of combining the FIDO2 technology with passwords managers
to provide a solution that works today,
and makes it easy to gradually move beyond passwords, as the support for FIDO2 increases.

Specifically, the thesis analyzes the capabilities of FIDO2 and KeePass password manager
(similar concepts should apply to other passwords managers as well), discusses the options
of replacing master passwords with FIDO2 devices, and provides a proof-of-concept implementation.
