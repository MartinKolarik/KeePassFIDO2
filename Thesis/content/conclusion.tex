As a first step, this thesis was meant to analyze the possibilities provided by the
KeePass plugin system and review the implementations of existing key provider plugins. This
analysis was covered in \autoref{section:keepass-password-manager} and could be summarized by
saying that KeePass has a flexible, well-documented plugin system, which allows implementing
alternative database unlock methods.

As a second step, we aimed to analyze the capabilities of FIDO2 and design a suitable approach
to implementing a key provider plugin utilizing a FIDO2 device, with a focus on security and ease of use.
We analyzed the FIDO2 specifications in \autoref{sec:fido2}, specified the key design goals on
a technical level in \autoref{sec:design-goals}, and then discussed four implementation approaches
to achieve them in \autoref{sec:implementation-options}.

We found that the first two implementation options, which would otherwise be the most suitable ones,
are not possible due to the limited capabilities of FIDO2 devices. Later, we found that the
third option would only work with some authenticators, which opted not to implement one of
the features of the FIDO2 specifications. This left us the fourth option as the only one
that should work with any device.

In \autoref{ch:implementation}, we fulfilled the final goal, which was implementing
the KeePass plugin. We have found, however, that due to restrictions on access to FIDO2
devices in the recent versions of Windows, the plugin will only work when running
under a privileged account. In \autoref{ch:testing}, we described test cases,
which can be used to verify the plugin functions correctly.

Even though the work meets all initial goals and was done in the "best possible" way,
there were many technical obstacles that forced us to use subpar solutions to some
problems, for the lack of better options. For that reason, we consider the most
notable result of this work to be the detailed analysis of the current options.

Future versions of FIDO specifications or new versions of the native Windows API may solve
some of the current issues and allow a better implementation approach.
